\chapter{Implementation}
In your implementation section, you can also use code listings to explain the details of your work.
To include the code you have two options: A) use the inline method for short one liner: \lstinline[language=Python]{[elem for elem in my\_list if elem.variable==filter_value]}, or
B) use multiline listings to visualize more complex examples.
Similar with figures, make sure to always reference your multiline listings and explain what they do.
In Listing~\ref{lst:test}, you find the function \lstinline[language=Python]{my_filter} with two parameters.
This function uses the list in the first parameter for all elements with the variable equal to the second parameter.

\begin{lstlisting}[language=Python,caption={A filter function in Python},label={lst:test}]
def my_filter(my_list, filter_value):
    new_list = []
    for elem in my_list:
        if elem.variable == filter_value:
            new_list.append(elem)
    return new_list
\end{lstlisting}

If you have even more detailed code listings, they belong into the appendix. 
For example, you can find a filter function with an additional sorting method in the Appendix~\ref{sec:app-lambda}.

In some cases, it might happen that your inline code does not fit into the current line anymore.
For example with this code here: \lstinline[language=Python]{[elem for elem in my\_list if elem.variable==filter_value]}
But don't worry, the arrow in the beginning of the line will highlight that!

If you require further details on the use of listings in LaTeX, you can find additional information on listings at the following two locations:
\begin{enumerate}
    \item \href{https://en.wikibooks.org/wiki/LaTeX/Source\_Code\_Listings}{https://en.wikibooks.org/wiki/LaTeX/Source\_Code\_Listings}
    \item \href{https://www.overleaf.com/learn/latex/Code\_listing}{https://www.overleaf.com/learn/latex/Code\_listing}
\end{enumerate}
