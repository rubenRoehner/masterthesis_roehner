\chapter*{Abstract}
As a first step towards your successful thesis, you need to evaluate a few things.
First, is the topic of value for you and do you feel motivated to work on this project for either three or six months?
This does not only mean that you work in a specific field for this period, but before you start, you already have a clear vision on how to start.
Of course, things change over time, and it would be boring, if you would already know everything in advance.
However, prepare yourself with research and discussions with your supervisor to be on top of the topic.
Second, you need to document your thesis in a way that \textbf{people can follow your thoughts and understand decisions} you had to take on the way.
This includes \textbf{clear writing} in a useful structure, where your supervisor will help you with.
But it also includes the choice of language: English or German? (others are not accepted)
Of course, English is preferred, as it fits best to the scientific environment where you will position yourself with this thesis.
But you can still choose German to create a document without loosing time if your English is not fluent or too stable.
In this document, you need to set the selected language at two different positions:
\sloppypar
\begin{enumerate}[label=\Alph*)]
    \item In the files \emph{template/title.tex} and \emph{template/metadata.tex}, you need to adjust the information on your thesis to generate a proper front page.
    \item In \emph{template/packages.tex} you need to set the correct option for the \emph{babel} package by either using \textbf{\textbackslash usepackage[english]\{babel\}} for english or \textbf{\textbackslash usepackage[ngerman]\{babel\}} for german.
    If you want to use umlauts with e.\,g. packages like hyphenat you should use a font encoding with good support of accents. To do this, put \textbf{\textbackslash usepackage[T1]\{fontenc\}} above the \textbf{\textbackslash usepackage[ngerman]\{babel\}} command.
    \item In \emph{thesis.tex} you need to change the strings starting with "List of" to german and add  \textbf{\textbackslash renewcommand\{\textbackslash listalgorithmname\}\{Algorithmusverzeichnis\}} above \textbf{\textbackslash addcontentsline\{toc\}\{chapter\}\{\textbackslash listalgorithmname\}}.
\end{enumerate}
