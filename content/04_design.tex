\chapter{Design}\label{sec:design}
The structure for the design chapter strongly depends on the topic of your thesis.
Throughout the process, you will refine this structure with your supervisor.

If you have a mathematical content in your thesis, you can use equations as follows.
Each equation uses its own space and should not be longer than one line, otherwise it is difficult to read.
For this document, the content of the equation is not important.
Still, we reference and explain it as an example.
Equation~\ref{eq:ifrnotas} defines the set of streams \sifr that will interfere with stream $s$ if no \gls{TAS} is configured.
For the purpose of consistent formulas, we introduce commands for complex terms in the file \emph{content/definitions.tex}

\begin{equation}\label{eq:ifrnotas}
    {\sifr} = \{ g | g \in \sedge \land g\neq s \land p_g \geq p_s \}
\end{equation}



In your thesis, you can also use more complex equation structures.
This structure still has one equation per line, but aligns them to be grouped together and indent the same way.
The alignment happens through the \& across the lines.
This complex structure still has references to each of the individual formulas.
Therefore, we can easily explain each of the Equations~\ref{eq:C1}, \ref{eq:C2}, and \ref{eq:combinetassync} in the text.

\begin{align}
    & \text{case C1: } {\tenqCT} < {\topen} \label{eq:C1}\\
    & \text{case C2: } {\tenqCT} + {\ddwell} + {\difrcross} < {\tclose} \label{eq:C2} \\
    & {\dgate} = 
        \begin{cases}
            {\topen} - {\tenqCT} + {\difrpath}         & \text{case C1} \\
            0                                                            & \text{case C2} \\
            {\ctgcl} - {\tenqCT} + {\topen} + {\difrpath}  & \text{otherwise} 
        \end{cases}\label{eq:combinetassync}
\end{align}


For your thesis, we recommend the use of tables to summarize series of data.
Sometimes, tables are easier to read than a lot of text with data.
In Table~\ref{tab:linkspeed}, we present the transmission delay of frames with different sizes at different link speeds.
Instead of writing all these numbers in a long paragraph, we can now use the time to highlight key insights.
For example, you can see that a frame with \SI{1522}{B} only consumes about twice the time at \SI{1}{Gbit/s}, compared to a frame with size \SI{64}{B} at \SI{100}{Mbit/s}.

\begin{table}
    \caption{Example transmission delays.}
    \label{tab:linkspeed}
    \begin{center}
        \begin{tabular}{r|cccc}
            \toprule
            \textbf{Size} & \textbf{100 Mbit/s} & \textbf{1 Gbit/s} & \textbf{2.5 Gbit/s} & \textbf{10 Gbit/s} \\
            \midrule
            \textbf{64 B} & 6.7~\textmu s & 672 ns & 269 ns & 67 ns \\
            \textbf{123 B} & 11.8~\textmu s & 1.2~\textmu s & 470 ns & 118 ns \\
            \textbf{1,522 B} & 123.4~\textmu s & 12.3~\textmu s & 4.9~\textmu s & 1.2~\textmu s \\
            \bottomrule
        \end{tabular}
    \end{center}
\end{table}



In your thesis, you might want to explain your solution with Pseudocode algorithms.
For example, Algorithm~\ref{alg:groupsort} presents a sorting algorithm for firewall rules.
In Line~\ref{lst:line:extendgroup}, we compare if a rule depends on the current group.

\begin{algorithm}
    \caption{Group Sort}\label{alg:groupsort}
    \begin{algorithmic}[1]
    \For{$x \in RuleSet$}
        \State $group \gets [x]$
        \For{$y \in RuleSet[x.idx:]$}
            \If{$dependency(group, y)$} 
                \Comment{Extend group}\label{lst:line:extendgroup}
                \State $group \gets group \# y$ 
            \Else
                \If{$hit\_count([y]) < hit\_count(group)$}
                    \State \Comment{Move complete group}
                    \State $y.idx \gets x.idx$
                    \For{$z \in group$}
                        \State $z.idx \gets z.idx + 1$
                    \EndFor
                \EndIf
                \State $group \gets [y]$
            \EndIf
        \EndFor
    \EndFor
\end{algorithmic}
\end{algorithm}




Finally, you can also use theorems, and lemmas in your thesis.
Please make sure (as always) to reference each of them in the text and explain their meaning.
For example, Theorem~\ref{thm:sometheorem}, or Lemma~\ref{thm:somelemma}.

\begin{theorem}\label{thm:sometheorem}
    Let \(f\) be a function whose derivative exists in every point, then \(f\) 
    is a continuous function.
\end{theorem}

\begin{lemma}\label{thm:somelemma}
    Given two line segments whose lengths are \(a\) and \(b\) respectively there is a 
    real number \(r\) such that \(b=ra\).
\end{lemma}

For formal procedures and algorithms, use the unified notation of proofs to state your conclusions.
An example for that follows here:

\begin{proof}\label{thm:someproof}
    To prove it by contradiction try and assume that the statement is false,
    proceed from there and at some point you will arrive to a contradiction.
\end{proof}